\documentclass[12pt,a4paper]{article}
\usepackage[utf8]{inputenc}
\usepackage[english]{babel}
\usepackage{natbib}
\usepackage{hyperref}
\usepackage{amsmath}
\usepackage{graphicx}
\usepackage{booktabs}
\usepackage{enumitem}
\usepackage{xcolor}
\usepackage{setspace}
\usepackage[left=2.5cm,right=2.5cm,top=2.5cm,bottom=2.5cm]{geometry}

\onehalfspacing

\title{Exploratory Computational Models Towards Understanding Human Decision-Making: Signal Choice and Group Structure in Norm Formation}
\author{Yifan Wu}
\date{}

\begin{document}

\maketitle

\section{Introduction}
\paragraph{Research Motivation:} Understanding the fundamental mechanisms of human decision-making remains a central challenge in cognitive and social sciences. This exploratory study examines computational models that may help illuminate basic human decision processes in social coordination contexts. By systematically varying decision strategies and communication structures, we aim to identify computational patterns that could inform future empirical research with human participants.

\paragraph{Research Question:} As a preliminary exploration towards understanding human decision-making fundamentals, this study examines how signal freedom and group size jointly shape coordination dynamics. We investigate: (1) the relationship between signal conditions (mandatory, optional, or none) and coordination success across varying group sizes, and (2) how different learning strategies (history-based versus reward-based) modify this relationship. These strategies represent simplified approximations of human cognitive processes—history-based modeling reflects declarative memory utilization, while reward-based approaches capture procedural learning mechanisms as proposed in dual-process theories of human cognition \citep{Daw2005}.

\section{Materials and Methods}
\paragraph{Experimental Design:} We implemented a binary-choice coordination game where agents select between two options (red or blue). Coordination success occurs when paired agents select identical options. We systematically varied group size (2, 3, 4, 6, 8, 10, 16, and 20 agents) to analyze coordination dynamics across different scales.

\paragraph{Agent Learning Strategies:} We implemented two learning strategies as simplified models of potential human cognitive processes:

\begin{enumerate}
    \item \textbf{History-Based Agents}: These agents maintain statistical records of past interaction outcomes and base decisions on pattern recognition, using accumulated historical data to inform current choices. 
    
    \item \textbf{Reward-Based Agents}: These agents implement reinforcement learning principles, modifying choice probabilities through immediate feedback after each interaction.
\end{enumerate}

\paragraph{Communication Conditions:} We implemented three communication scenarios:
\begin{enumerate}
    \item \textbf{No Signal}: Agents make choices without prior communication.
    
    \item \textbf{Mandatory Signal}: All agents must communicate their intended choice before acting.
    
    \item \textbf{Optional Signal}: Agents can choose whether to signal their intentions, creating information asymmetries.
\end{enumerate}

\paragraph{Measurement and Analysis:} Using a rotation-based matching system that systematically pairs all agents over time, we tracked three metrics across 100,000 possible interaction rounds:
\begin{enumerate}
    \item \textbf{Coordination speed}: Rounds required to reach convergence on a shared norm
    \item \textbf{Success rate}: Proportion of groups achieving coordination within the time limit
    \item \textbf{Norm selection}: Distribution of final coordinated choices (blue vs. red)
\end{enumerate}

\section{Results}

\paragraph{Finding 1: Group Size Effects} 
Group size effects on coordination varied systematically based on learning strategy implementation. In small groups (2-4 agents), both strategies achieved coordination within 100 rounds. As group size increased, performance divergence became quantitatively significant.

History-based agents exhibited exponential increases in coordination time with larger groups, reaching 100,000 rounds without convergence in 20-agent groups under both no-signal and mandatory signal conditions (0\% success rate). Reward-based agents maintained consistent coordination efficiency in large groups, achieving convergence in 372 rounds with mandatory signals in 20-agent groups—a 250-fold efficiency difference compared to history-based counterparts.

\paragraph{Finding 2: Communication Impact} 
Communication structure effects interacted with both group size and learning strategy variables. Optional signaling improved history-based agents' performance in larger groups. With 16 agents, optional signaling reduced convergence time by 88\% (11,363 vs. 95,783 rounds) compared to no signaling conditions.

Reward-based agents performed optimally with mandatory signaling across all group sizes, achieving coordination in 69 rounds versus 141 with no signals in 8-agent groups.

\paragraph{Finding 3: Success Rates} 
Coordination reliability measurements revealed systematic strategy differences. Reward-based agents achieved 100\% coordination success across all tested group sizes and communication conditions. History-based agents exhibited declining success rates as group size increased, reaching 0\% success in 20-agent groups under both no-signal and mandatory-signal conditions.

In large-group environments (20 agents), optional signaling enabled history-based agents to maintain a 70\% success rate.

\paragraph{Finding 4: Norm Selection Patterns} 
Final norm selection (blue vs. red convergence) showed no systematic bias across conditions. The blue choice convergence rate ranged from 0.25 to 0.70 across different scenarios, with no clear pattern related to group size, signal condition, or learning strategy.

\section{Discussion}

\paragraph{Implications for Human Decision-Making Models:} 
This exploratory study offers preliminary insights into potential computational mechanisms that may underlie human coordination behavior. The stark performance differences between reward-based and history-based learning strategies suggest that human coordination might employ multiple cognitive systems with varying effectiveness depending on context.

\paragraph{Adaptation Capacity and Coordination:} 
The data demonstrate a clear relationship between adaptation capacity and coordination efficiency. Reward-based learning exhibits superior coordination in complex environments, while history-based approaches show limitations when updating established patterns. These differences may parallel human decision-making under varying cognitive loads and informational complexity.

\paragraph{Signal Flexibility Effects:} 
Optional signaling consistently improved coordination for history-based agents in larger groups. With 10 agents, optional signaling reduced convergence time by 96\% compared to mandatory signaling conditions (682.5 vs. 45,062.5 rounds) for history-based agents.

This effect was not observed for reward-based agents, where mandatory signaling provided optimal coordination efficiency.

\section{Conclusion}

\paragraph{Key Findings as a Foundation for Human Decision Models:} 
This exploratory research demonstrates potential computational mechanisms that might underlie human coordination behavior. By systematically varying learning strategies and communication structures across different group sizes, we have identified interaction patterns that may inform models of human decision-making in coordination contexts. Three primary findings emerge:

\begin{enumerate}
    \item Learning strategy significantly impacts coordination efficiency, with reward-based learning providing a 250-fold coordination advantage in large groups.
    
    \item Optional signaling creates coordination pathways that partially compensate for limited adaptation capacity, enabling a 70\% success rate for history-based agents in large groups compared to 0\% under fixed signaling conditions.
    
    \item Coordination performance depends on the match between learning mechanisms and communication structures.
\end{enumerate}

\paragraph{Limitations and Future Directions:} 
As an exploratory computational study, this research has inherent limitations in directly characterizing human behavior. Future research should address these limitations through:

\begin{enumerate}
    \item \textbf{Human validation studies}: Conducting parallel human experiments to calibrate computational models against actual behavioral data
    
    \item \textbf{Hybrid model development}: Creating integrated models incorporating both history-based and reward-based components to better approximate human cognitive processes
\end{enumerate}

\bibliographystyle{apalike}
\begin{thebibliography}{10}

\bibitem[Balliet(2010)]{Balliet2010}
Balliet, D. (2010).
\newblock Communication and cooperation in social dilemmas: A meta-analytic review.
\newblock \emph{Journal of Conflict Resolution, 54}(1), 39--57.

\bibitem[Barcelo and Capraro(2015)]{BarceloCapraro2015}
Barcelo, H., \& Capraro, V. (2015).
\newblock Group size effect on cooperation in one-shot social dilemmas.
\newblock \emph{Scientific Reports, 5}, 7937.

\bibitem[Baronchelli(2018)]{Baronchelli2018}
Baronchelli, A. (2018).
\newblock The emergence of consensus: A primer.
\newblock \emph{Royal Society Open Science, 5}(2), 172189.

\bibitem[Centola and Baronchelli(2015)]{Centola2015}
Centola, D., \& Baronchelli, A. (2015).
\newblock The spontaneous emergence of conventions: An experimental study of cultural evolution.
\newblock \emph{Proceedings of the National Academy of Sciences, 112}(7), 1989--1994.

\bibitem[Claus and Boutilier(1998)]{Claus1998}
Claus, B., \& Boutilier, C. (1998).
\newblock The dynamics of reinforcement learning in cooperative multiagent systems.
\newblock \emph{AAAI/IAAI}, 746--752.

\bibitem[Daw et al.(2005)]{Daw2005}
Daw, N. D., Niv, Y., \& Dayan, P. (2005).
\newblock Uncertainty-based competition between prefrontal and dorsolateral striatal systems for behavioral control.
\newblock \emph{Nature Neuroscience, 8}(12), 1704--1711.

\bibitem[Flache et al.(2017)]{Flache2017}
Flache, A., Mäs, M., Feliciani, T., Chattoe-Brown, E., Deffuant, G., Huet, S., \& Lorenz, J. (2017).
\newblock Models of social influence: Towards the next frontiers.
\newblock \emph{Journal of Artificial Societies and Social Simulation, 20}(4), 2.

\end{thebibliography}

\end{document} 